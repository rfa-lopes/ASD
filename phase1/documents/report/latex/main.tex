%%
%% This is file `sample-sigconf.tex',
%% generated with the docstrip utility.
%%
%% The original source files were:
%%
%% samples.dtx  (with options: `sigconf')
%% 
%% IMPORTANT NOTICE:
%% 
%% For the copyright see the source file.
%% 
%% Any modified versions of this file must be renamed
%% with new filenames distinct from sample-sigconf.tex.
%% 
%% For distribution of the original source see the terms
%% for copying and modification in the file samples.dtx.
%% 
%% This generated file may be distributed as long as the
%% original source files, as listed above, are part of the
%% same distribution. (The sources need not necessarily be
%% in the same archive or directory.)
%%
%% The first command in your LaTeX source must be the \documentclass command.
\documentclass[sigconf]{acmart}

%%
%% \BibTeX command to typeset BibTeX logo in the docs
\AtBeginDocument{%
  \providecommand\BibTeX{{%
    \normalfont B\kern-0.5em{\scshape i\kern-0.25em b}\kern-0.8em\TeX}}}

%% Rights management information.  This information is sent to you
%% when you complete the rights form.  These commands have SAMPLE
%% values in them; it is your responsibility as an author to replace
%% the commands and values with those provided to you when you
%% complete the rights form.
\setcopyright{acmcopyright}
\copyrightyear{2020}
\acmYear{2020}
%\acmDOI{10.1145/1122445.1122456}

%% These commands are for a PROCEEDINGS abstract or paper.
\acmConference[ASD20/21]{The first project delivery of ASD2021}{2020}{FCT, NOVA University of Lisbon, Portugal}
\acmBooktitle{The Projects of ASD - first delivery, 2020, Faculdade de Ciências e Tecnologia, NOVA University of Lisbon, Portugal}
%\acmPrice{15.00}
%\acmISBN{978-1-4503-XXXX-X/18/06}


%%
%% Submission ID.
%% Use this when submitting an article to a sponsored event. You'll
%% receive a unique submission ID from the organizers
%% of the event, and this ID should be used as the parameter to this command.
%%\acmSubmissionID{123-A56-BU3}

%%
%% The majority of ACM publications use numbered citations and
%% references.  The command \citestyle{authoryear} switches to the
%% "author year" style.
%%
%% If you are preparing content for an event
%% sponsored by ACM SIGGRAPH, you must use the "author year" style of
%% citations and references.
%% Uncommenting
%% the next command will enable that style.
%%\citestyle{acmauthoryear}

%%..........................MEUS PACOTES................................
\usepackage[utf8]{inputenc}
\usepackage{amsmath}
\usepackage{amsfonts}
\usepackage{graphicx}
\usepackage[colorinlistoftodos]{todonotes}
\usepackage{algorithm}
\usepackage{algpseudocode}
\usepackage[toc,page]{appendix}
\usepackage{listings}
\lstset{frame=tb,
  language=Java,
  aboveskip=3mm,
  belowskip=3mm,
  showstringspaces=false,
  columns=flexible,
  basicstyle={\small\ttfamily},
  numbers=none,
  breaklines=true,
  breakatwhitespace=true,
  tabsize=1
}

\usepackage{geometry}
 \geometry{
 a4paper,
 total={210mm,297mm},
 left=20mm,
 right=20mm,
 top=20mm,
 bottom=20mm,
 }
%%..........................MEUS PACOTES................................

%%
%% end of the preamble, start of the body of the document source.
\begin{document}

%%
%% The "title" command has an optional parameter,
%% allowing the author to define a "short title" to be used in page headers.
\title{Implementation and Experimental Measurements of Membership and Broadcast Algorithms}

%%
%% The "author" command and its associated commands are used to define
%% the authors and their affiliations.
%% Of note is the shared affiliation of the first two authors, and the
%% "authornote" and "authornotemark" commands
%% used to denote shared contribution to the research.
\author{Rodrigo Faria Lopes}
\authornote{Student number 50435. Rodrigo foi responsável pela implementação do HyParView e escrita dos scripts para a facilitação de testes automáticos no Cluster.}
\email{rfa.lopes@campus.fct.unl.pt}
\affiliation{%
  \institution{MIEI, DI, FCT, UNL}
}

\author{Miguel Candeias}
\authornote{Student number 50647. Miguel foi o responsável pela implementação do Cyclon e do Eager Push, assim como do script em python para analizar os logs para gerar as estatísticas experimentais.}
\email{mb.candeias@campus.fct.unl.pt}
\affiliation{%
  \institution{MIEI, DI, FCT, UNL}
}

\author{Salvador Rosa Mendes}
\authornote{Student number 50503. Salvador desenvolveu a implementação do PlumTree.}
\email{sr.mendes@campus.fct.unl.pt}
\affiliation{%
  \institution{MIEI, DI, FCT, UNL}
}

%%
%% By default, the full list of authors will be used in the page
%% headers. Often, this list is too long, and will overlap
%% other information printed in the page headers. This command allows
%% the author to define a more concise list
%% of authors' names for this purpose.
\renewcommand{\shortauthors}{Lopes, Candeias, and Mendes.}

%%
%% The abstract is a short summary of the work to be presented in the
%% article.
\begin{abstract}
  %TODO: RODRIGO
  %Primeiro paragrafo - Contexto e objectivos
  %Segundo paragrafo - O que fizemos em concreto e resultados obtidos
  Este artigo tem como objetivo a implementação e comparação de alguns algoritmos de Bradcast e Membership que servirão como base de uma aplicação previamente fornecida. O foco deste artigo é testar como funcionam todas as combinações dos algoritmos propostos e avaliar os seus resultados. Os algoritmos que iremos abordar são: HyParView \cite{hyparview} e Cyclon \cite{cyclon} como algoritmos de \textit{Partial Mermbership}, e os algoritmos Plumtree \cite{plumtree} e Eager \cite{hyparview} como algoritmos de \textit{Broadcast}.
  
  O trabalho desenvolvido começou pelo estudo e implementação dos algoritmos de forma individual utilizando a biblioteca Babel \cite{babel}. Posteriormente foram desenvolvidos testes para cada par de algoritmos (Broadcast + Membership) variando várias parâmetros, tais como o tamanho do payload e a taxa de transmissão. Por fim, os resultados obtidos serão discutidos e avaliados de acordo com os nossos resultados.
\end{abstract}


%%
%% This command processes the author and affiliation and title
%% information and builds the first part of the formatted document.
\maketitle

\section{Introduction}
TODO
(4 ou 5 paragrafos)
Contexto - Porque é que isto é interessante?
Contexto muito geral (paragrafo)
Contexto menos geral (paragrafo)
Contexto especifico (paragrafo)

(paragrafo)
Estrutura do documento

\section{Related Work}
TODO - Explicar os algoritmos

2 Subseccoes
    1-Overlay protocols
    2-Broadcast gossip
    3(opcional) - Discutir outros estudos praticos destas soluções (poque é que o nosso é melhor?)

\section{Implementation}
%TODO - Explicar a implementação e possíveis modificações
%pseudocodigo, etc

%---------------------------------------------

%TODO: Indice sobre os topicos que vao ser abordados
Nesta secção vamos falar sobre a implementação dos algoritmos. Decidimos dividir esta secção em duas partes de forma semelhante à secção 2, de modo a separar as componentes do projeto.

Na secção 3.1 iremos abordar a implementação dos algoritmos de disseminação (Eager Push e Plumtree), e na secção seguinte, 3.2, serão abordados os algoritmos de \textit{Partial Membership} (HyParView e Cyclon).

\subsection{Epidemic Dissemination Protocols (Broadcast)}
%TODO: Breve introdução dos protocolos

\subsubsection{Eager Push Gossip}
%TODO: Descrição geral
%TODO: Explicação do protocolo e explicação de possíveis modificações
Através do material leccionado nas aulas, e da dissertação \cite{leitao}, foi possível estudar e implementar o algoritmo.

Este algoritmo é do tipo \textit{gossip}, o que permite não sub carregar o emissor e ao mesmo também condensa menos a \textit{overlay network }quando é necessário enviar mensagens. Através do calculo do parâmetro de \textit{fanout}, cujo o seu valor é o logaritmo do numero actual de vizinhos, é possível assim enviar uma mensagem cujo a probabilidade de chegar a todos os nós no \textit{overlay} é bastante alta. 

A função Init deste protocolo inicializa o estado do processo, e então aguarda pela notificação do protocolo de \textit{membership} para que se possa utilizar o canal. Na função \textit{EagerMessage} é possível então verificar a aplicação prática do envio apenas para um numero reduzido de nós a mensagem que se pretende disseminar na rede.

De modo a ilustrar melhor o comportamento do algoritmo, de seguida apresenta-se o pseudocódigo do mesmo (Algoritmo 1).

\begin{algorithm}
   \caption{Eager Push Broadcast}
    \begin{algorithmic}[1]
      \Function{Init}{myIp}
        \State myself $\leftarrow$ myIp
        \State t $\leftarrow$ $\bot$
        \State neigh $\leftarrow$ \{ \}
        \State delivered $\leftarrow$ \{ \}
        \State channelReady $\leftarrow$ False
	  \EndFunction
    \paragraph{}
    
\Function{EagerMessage}{GossipMessage}
     \If{GossipMessage $\notin$ delivered}
         \If{GossipMessage.ttl > -1}
         \State \textbf{Trigger }DeliverNotifications()
          \State neigh $\leftarrow$ getNeigh()
          \State t $\leftarrow$ $\ln$(\#neigh)
           \If{\#neighbors > 0}
                gossipTargets $\leftarrow$ RandomSelection(t)
                
                 \For{p $\in$ gossipTargets}
                    \State \textbf{Trigger} Send(GossipMessage, p);
                    \EndFor
            \EndIf
        \EndIf
        \EndIf
\EndFunction

\paragraph{}

\Function{BroadcastRequest}{BroadcastRequest}
    \If{channelReady}
        \State gossipMessage $\leftarrow$ GossipMessage(request.MsgId, request.Sender, sourceProtocol. request.Payload, ttl)
        \State \textbf{Trigger} EagerMessage(gossipMessage)
        \EndIf
\EndFunction
	  
\end{algorithmic}
\end{algorithm}

\subsubsection{Plumtree}
%TODO: Descrição geral
%TODO: Explicação do protocolo e explicação de possíveis modificações

O Plumtree é uma implementação do tipo \textit{gossip} e é uma junção parcial dos algoritmos de \textit{gossip Eager Push} e \textit{Lazy Push}, sendo assim mais equilibrado em termos de recursos e performance.
No algoritmo é feita uma divisão dos seus \textit{peers}, os \textit{Eager Push Peers} e os \textit{Lazy Push Peers}. 
Os \textit{Eager Push Peers} são todos os nós que ainda não enviaram uma mensagem repetida e que portanto são beneficiados se for mantida uma comunicação bi-direcional entre eles, os nós pertencentes a este grupo comunicaram posteriormente através de uma forma do algoritmo Eager Push, que consiste em enviar a mensagem recebida para todos os \textit{peers} que pertençam a este grupo.
Os \textit{Lazy Peers} são nós que devido a terem enviado uma mensagem repetida foram removidos dos \textit{Eager Push Peers} e enviados por parte do receptor da mensagem repetida uma mensagem \textit{Prune} que lhe indica que a mensagem era repetida e que portanto podem remover o nó receptor dos \textit{Eager Push Peers} mantendo-se assim sincronizados. Os nós pertencentes a este grupo serão enviados mais tarde através de uma forma do algoritmo \textit{Lazy Push} uma mensagem, mensagem \textit{Prune},  que apenas contém o ID da mensagem que o emissor possui, estas mensagens não têm obrigatoriamente de sair imediatamente após o nó ser movido para os \textit{Lazy Peers} e por isso de forma a optimizar o processo são enviadas em batches através de um \textit{dispatch protocol}.

O algoritmo implementado e usado nos testes das secções seguintes, foi passado para pseudocódigo (Algoritmo 2) de modo a permitir uma melhor percepção de como este funciona.

\begin{algorithm}
   \caption{PlumTree}
    \begin{algorithmic}[1]
      \Function{Init}{}
        \State eagerPushPeers $\leftarrow$ \{ \}
        \State lazyPushPeers $\leftarrow$ \{ \}
        \State lazyQueue $\leftarrow$ \{ \}
        \State received $\leftarrow$ \{ \}
        \State channelReady $\leftarrow$ False
	  \EndFunction
    \paragraph{}
    
\Function{BroadcastRequest}{Request, sourceProtocol}
     \If{channelReady}
        \State gossipMessage $\leftarrow$ GossipMessage(request.MsgId, request.Sender, sourceProtocol. request.Payload, ttl)
        \State \textbf{Call} eagerPush(gossipMessage, myself, currentProtocolId, channelId)
         \State lazyPush(gossipMessage, currentProtocolId)
         \State \textbf{trigger} DeliverNotification
         \State received $\leftarrow$ $\cup$ request.MsgId
        \EndIf
\EndFunction

    \paragraph{}


\paragraph{}

\Function{receive}{ProtoMessage, HostFrom, sourceProto, channelId}
    \If{type(ProtoMessage) = "LazyMessage"}
        \State \textbf{Call} receive(ProtoMessage, HostFrom, sourceProto, channelId) //Prune Msg
    \Else
        \State \textbf{Call} receive(ProtoMessage, HostFrom, sourceProto, channelId)
    \EndIf
\EndFunction

    \paragraph{}

\Function{dispatch}{}
   \For{lazyMessage $\in$ LazyQueue}
    \State \textbf{trigger} Send(lazyMessage, lazyMessage.Destination)
    \State LazyQueue $\leftarrow$ LazyQueue  $\setminus$ lazyMessage 
    \EndFor
\EndFunction

    \paragraph{}


\Function{eagerPush}{GossipMessage, Host, sourceProtocol, channelId}
     \For{h $\in$ LazyQueue eagerPushPeers $\land$ h $\ne$ myself}
     \State Send(GossipMessage, h) $\cup$ LazyMessage(h, message.MsgId, HostMyself,sourceProtocol)
     \State \textbf{call} dispatch()
     \EndFor
   
\EndFunction

    \paragraph{}


\Function{lazyPush}{GossipMessage, sourceProtocol)}
    \For{h $\in$ LazyQueue eagerPushPeers $\land$ h $\ne$ myself}
    \State lazyQueue $\leftarrow$ lazyQueue
    \EndFor
\EndFunction
	  
\end{algorithmic}
\end{algorithm}


%---------------------------------------------

\subsection{Partial Membership Protocols (Unstructed Overlay)}
%TODO: Breve introdução dos protocolos

\subsubsection{HyParView}
%TODO: RODRIGO
%TODO: Descrição geral
%TODO: Explicação do protocolo e explicação de possíveis modificações
A implementação deste algoritmo foi totalmente inspirada em \cite{hyparview}, tendo este sido implementado de acordo com as instruções do artigo referido.

Inicialmente o protocolo inicia com uma mensagem de \textit{Join} a um elemento aleatório da lista de contactos do novo nó que se quer juntar à rede. O envio desta mensagem tem origem na função \textit{Init} onde também é lá que se inicializam os temporizadores (JoinTimer e ShuffleTimer). Após o envio da mensagem de \textit{Join} o processo espera, durante um período de tempo, por uma mensagem de \textit{JoinReply}, caso a mensagem chegue dentro do tempo definido no \textit{JoinTimer}, este já faz parte da rede e continua o algoritmo normalmente como definido em \cite{hyparview}, desligando o \textit{JoinTimer}. Caso contrário, isto é, o tempo definido para o \textit{JoinTimer} passou, é lançado um processo de selecção de um novo contacto de forma aleatória. Com este mecanismo, caso o novo nó apenas tenha um contacto, este ficará num ciclo infinito à espera que este aceite o sei pedido de \textit{Join}, ao contrário do algoritmo definido em \cite{hyparview} onde caso o contacto não existisse ou estivesse em baixo, este processo terminava. Todo este novo mecanismo de \textit{Join} de um novo nó está presente no Algoritmo 1 HyParView - Parte 1.

Algumas verificações adicionais poderiam ser implementadas para mitigar alguns dos problemas desta solução, algumas delas irão ser referidas na secção 5 como \textit{future work}.

Após o protocolo de \textit{Join} o algoritmo continua o seu processo de adesão do novo nó, espalhando mensagens à vizinhança do contacto de modo a estes saibam da existência de um novo nó na rede.

Todo o algoritmo implementado e usado nos testes das secções seguintes, foi passado para pseudocódigo (Algoritmos 3 a 6) de modo a permitir uma melhor percepção de como este funciona.

\begin{algorithm}
   \caption{HyparView- Parte 1}
    \begin{algorithmic}[1]
      \Function{Init}{}
        \State n $\leftarrow$ n $\in$ contactsNodes
        \State Setup Periodic Timer ( JoinTimer, t1 )
	    \State Setup Periodic Timer ( ShuffleTimer, t2 )
	    \State Send( Join, n, myself )
	  \EndFunction
	    \State
	  \Function{Receive}{Join, newNode}
        \If {isfull( activeView )}
            \State \textbf{trigger} dropRandomElementFromActiveView
        \EndIf
        \State \textbf{trigger} addNodeActiveView(newNode)
        \State Send(JoinReply, newNode, myself)
        \For{n $\in$ activeView and n $\ne$ newNode}
            \State Send(ForwardJoin, n, newNode, ARWL, myself)
        \EndFor
     \EndFunction
     \State
     \Function{Receive}{JoinReply, contactNode}
        \State \textbf{trigger} addNodeActiveView(newNode)
        \State Cancel Periodic Timer(JoinTimer, t)
     \EndFunction
     \State
     \Function{Receive}{ForwardJoin, newNode, ttl, sender}
        \If {ttl==0 || \#activeView==0}
            \State \textbf{trigger} addNodeActiveView(newNode)
        \Else
            \If {ttl==PRWL}
                \State \textbf{trigger} addNodePassiveView(newNode)
            \EndIf
            \State n $\leftarrow$ n $\in$ activeView and n $\ne$ sender
            \State Send(ForwardJoin, n, newNode, ttl-1, myself)
        \EndIf
     \EndFunction
     \State
     \Function{dropRandomElementFromActiveView}{}
        \State n $\leftarrow$ n $\in$ activeView
        \State Send(Disconnect, n, myself)
        \State activeView $\leftarrow$ activeView $\setminus$ $\{$n$\}$
        \State passiveView $\leftarrow$ passiveView $\cup$ $\{$n$\}$
     \EndFunction
     \State
     \Function{addNodeActiveView}{node}
        \If {node $\ne$ myself and node $\notin$ activeView}
            \If {isfull( activeView )}
                \State \textbf{trigger} dropRandomElementFromActiveView
            \EndIf
            \State activeView $\leftarrow$ activeView $\cup$ $\{$n$\}$
        \EndIf
     \EndFunction
     \State
     \Function{addNodePassiveView}{node}
        \If {node $\ne$ myself and node $\notin$ activeView and node $\notin$ passiveView}
            \If {isfull( passiveView )}
                \State n $\leftarrow$ n $\in$ passiveView
                \State passiveView $\leftarrow$ passiveView $\setminus$ $\{$n$\}$
            \EndIf
            \State passiveView $\leftarrow$ passiveView $\cup$ $\{$node$\}$
        \EndIf
     \EndFunction
    
\end{algorithmic}
\end{algorithm}
%--------------------------Continuação (nao cabia na página)
\begin{algorithm}
   \caption{HyParView - Parte 2}
    \begin{algorithmic}[1]
     \Function{Receive}{Disconnect, peer}
        \If {peer $\in$ activeView}
            \State activeView $\leftarrow$ activeView $\setminus$ $\{$peer$\}$
            \State \textbf{trigger} addNodePassiveView(peer)
        \EndIf
     \EndFunction
     \State
    \Function{Receive}{Neighbor, peer, isPriority}
        \If {isPriority}
            \State \textbf{trigger} addNodeActiveView(peer)
        \Else
            \If {isfull(activeView)}
                \State Send(Reject, peer, myself)
            \Else
                \State passiveView $\leftarrow$ passiveView $\setminus$ $\{$peer$\}$
			    \State \textbf{trigger} addNodeActiveView(peer)
            \EndIf
        \EndIf
     \EndFunction
     \State
     \Function{Receive}{Reject, peer}
        \State passiveView $\leftarrow$ passiveView $\setminus$ $\{$peer$\}$
        \State \textbf{trigger} attemptPassiveViewConnection
        \State passiveView $\leftarrow$ passiveView $\cup$ $\{$peer$\}$
     \EndFunction
     \State
     \Function{Receive}{Shuffle, ttl, passiveViewSample, activeViewSample, peer}
        \If {ttl - 1 == 0 and \#activeView > 1}
            \State n $\leftarrow$ n $\in$ activeView and n $\ne$ peer
            \State Send(Shuffle, ttl - 1, n, myself)
        \Else
            \State sample $\leftarrow$ \textbf{trigger} getSample(passiveView, \#passiveViewSample)
            \State Send(ShuffleReply, sample, peer, myself)
        \EndIf
        \State sample $\leftarrow$ passiveViewSample $\cup$ activeViewSample
	    \State \textbf{trigger} integrateElementsIntoPassiveView(samples)
     \EndFunction
     \State
     \Function{Receive}{ShuffleReply, sample, peer}
        \State \textbf{trigger} integrateElementsIntoPassiveView(samples)
     \EndFunction
     
     \Function{integrateElementsIntoPassiveView}{sample}
        \State sample $\leftarrow$ sample $\setminus$ $\{$activeView $\cup$ passiveView $\cup$ $\{$myself$\}$ $\}$
        \If {\#passiveView + \#sample > passiveViewMaxSize}
            \State \textbf{trigger} removeRandomElements(passiveView, \#sample)
        \EndIf
        \State passiveView $\leftarrow$ passiveView $\cup$ sample
     \EndFunction
     \State
     \Function{JoinTimer}{}
        \State n $\leftarrow$ n $\in$ contactsNodes
        \State Send(Join, n, myself)
     \EndFunction
     
\end{algorithmic}
\end{algorithm}

%--------------------------Continuação (nao cabia na página)
\begin{algorithm}
   \caption{HyParView - Parte 3}
    \begin{algorithmic}[1]
    
    \Function{attemptPassiveViewConnection}{}
        \If {\#passiveView > 0}
            \If {\#activeView == 0}
                \State isPriority = true
            \Else
                \State isPriority = false
            \EndIf
            \State n $\leftarrow$ n $\in$ passiveView
            \State passiveView $\leftarrow$ passiveView $\setminus$ $\{$n$\}$
            \State \textbf{trigger} addNodeActiveView(newNode)
            \State Send(Neighbor, isPriority, n, myself)
        \EndIf
     \EndFunction
     \State
    \Function{ShuffleTimer}{}
        \If{\#activeView > 0}
            \State aSample $\leftarrow$ \textbf{trigger} getSample(activeView, KA)
            \State pSample $\leftarrow$ \textbf{trigger} getSample(passiveView, KP)
            \State n $\leftarrow$ n $\in$ activeView
            \State Send(Shuffle, aSample, pSample, n, myself)
        \EndIf
     \EndFunction
     \State
     \Function{getSample}{set, size}
        \State retult $\leftarrow$ $\{$ $\}$
        \For{i=0 \textbf{to} size-1}
            \State n $\leftarrow$ $\{$n$\}$ $\in$ set
            \State result $\leftarrow$ $\{$result$\}$ $\cup$ n
        \EndFor
     \EndFunction
    
\end{algorithmic}
\end{algorithm}


\subsubsection{Cyclon}
%TODO: Descrição geral
%TODO: Explicação do protocolo e explicação de possíveis modificações

A implementação deste algoritmo, foi baseada no material de apoio do Cyclon que foi leccionado na cadeira, e também respeitando algumas propriedades encontradas na dissertação \cite{leitao} e finalmente o artigo \cite{cyclon}

O primeiro passo de execução do algoritmo, consiste em, verificar se quando o processo que se pretende juntar ao \textit{overlay} tem algum nó conhecido dentro deste, então este será a o primeiro vizinho do mais recente processo. Este processo ocorre na função init. Como explicado na secção anterior, nesta implementação do protocolo decidimos não validar se o contacto conhecido dentro do \textit{overlay} ainda existia ou não, isto deve-se ao facto de como o protocolo recorre a um procedimento periódico, que é este o responsável pela manutenção do \textit{overlay} em termos de frescura de de processos validos, como irá ser explicado mais adiante, garantindo assim que o contacto invalido será esquecido.

Como dito anteriormente, a função de \textit{shuffle }que ocorre periodicamente é a responsável pela manutenção do \textit{overlay}, inicialmente esta função incrementa a idade de todos os processos que se encontram na vizinhança de um nó, após este incremento, é então eleito o nó mais antigo dentro da vizinhança de um processo, e temporariamente  retira-se o mesmo e procede-se à tentativa de comunicação com este.

A função apresentada na linha 29 no pseudo codigo do algoritmo, é chamada quando um nó vizinho do processo, pede uma amostra da vizinhança actual para poder actualizar a sua. Esta função assim como a que está representada na linha 26, terminam sempre numa chamada a um procedimento chamado \textit{"Merge Views"}. Este procedimento fundamental do algoritmo, sendo este o responsável pela actualização da nova vizinhança do processo onde este procedimento foi despoletado, este consiste em analisar para cada nó na vizinhança que foi enviada por um vizinho, se o nó já existe na vizinhança, caso exista e este tiver uma \textit{timestamp} mais recente, actualiza-se a vizinhança. Se o nó não existir e houver espaço na vizinhança, apenas se insere o nó nesta, caso contrario tenta-se escolher um nó que esteja contido na vizinhança recebida e na vizinhança do nó onde está a correr o procedimento, caso não seja possível, então finalmente escolhe-se um nó da vizinhança onde está a correr o procedimento, retira-se o mesmo, e insere-se o nó da vizinhança recebida.

De seguida apresenta-se o pseudocódigo do algoritmo, de maneira a ilustrar o funcionamento do mesmo (algoritmo 4).


\begin{algorithm}
   \caption{Cyclon}
    \begin{algorithmic}[1]
      \Function{Init}{contactNode, timer}
        \State \textbf{trigger} CreateNotification(channelId)
        \If{contactNode $\ne$ $\bot$}
            \State neigh $\leftarrow$ neingh $\cup$ \{ \}
        \EndIf
        
        \State Setup Periodic Timer Shuffle(timer)
	  \EndFunction
	          \paragraph{}

	  \Function{Shuffle}{}
	   \If{\#neigh > 0}
	    \For{(p, age) $\in$ neigh}
            \State neigh $\leftarrow$ neingh $\setminus$ (p,age) $\cup$ (p,age +1)

        \EndFor
            \State  oldest $\leftarrow$ GetOldest(neigh)
            \State  neigh $\leftarrow$ neigh $\setminus$ oldest
            \State sample $\leftarrow$ randomSubset(neigh)
            \State \textbf{Trigger} Send (ShuffleRequest,p,sample $\cup$ \{(myself,0) \})
        \EndIf
        \EndFunction
        \paragraph{}
        \Function{Receive}{ShuffleReply, s, PeerSample}
            \State Call mergeViews(peerSample, sample)
        \EndFunction
                \paragraph{}

         \Function{Receive}{ShuffleRequest, s, PeerSample}
            \State temporarySample $\leftarrow$ temporarySample $\cup$ neigh
            \State \textbf{Trigger }Send(ShuffleReply, s, temporarySample)
            \State Call mergeViews(peerSample,temporarySample)
        \EndFunction
                \paragraph{}

        \Function{Merge Views}{peerSample, mySample}
           \For{(p, age) $\in$ peerSample}
            \If{(p', age') $\in$ neigh $\land$ p = p'}
                \If{age'> age}
                    \State neigh $\leftarrow$ (neigh $\setminus$(p', age')) $\cup$ \{(p, age) \})  \EndIf
                \Else
                    \If{\#neigh< sampleSize}
                        \State neigh $\leftarrow$ neigh $\cup$ \{(p, age)\})  
                    
                     \Else
                        \State (x,age’):(x,age') $\cup$ neigh $\land$ (x, age'') $\cup$ mySample
                        \If{\#neigh< sampleSize}
                            \State (x, age') $\leftarrow$ (x,age’):(x, age’) $\cup$ neigh
                            \EndIf
                        \State neigh $\leftarrow$ (neigh $\setminus$ (x, age')) $\cup$ \{(p, age)\})
                    \EndIf
                \EndIf
                    
                   
                        
                   
            \EndFor
        \EndFunction
	  
\end{algorithmic}
\end{algorithm}


\section{Experimental Evaluation}
TODO

Detalhes de forma a que se alguem quiser replicar os testes, consiga:
-Tipo de hardware(caractristicas)
-parametros dos protocolos
-Tempos
-Processamento às medidas?

metodologia

resultados (gráficos melhor que tabelas)
gráfico único com várias linhas

discussão dos resultados um a um
-Explicar o que esta no gráfico
-Extrair conclusões

\section{Conclusions}
TODO

sumário do que apresentamos

future work

limitações que o nosso trabalho tem

\bibliographystyle{ACM-Reference-Format}

\begin{thebibliography}{9}

%%%%%%%%%%%%%%%%%%%%%%%%%%%%%%
\bibitem{hyparview} 
João Leitão, José Pereira, and Luís Rodrigues. 
\textit{HyParView}: a membership protocol for reliable gossip-based broadcast.
\\\texttt{https://asc.di.fct.unl.pt/~jleitao/pdf/dsn07-leitao.pdf}
%%%%%%%%%%%%%%%%%%%%%%%%%%%%%%
\bibitem{cyclon} 
Spyros Voulgaris, Daniela Gavidia, and Maarten van Steen. 
\textit{Cyclon}: Inexpensive membership management for unstructured p2p overlays.
\\\texttt{http://citeseerx.ist.psu.edu/viewdoc/summary?doi=10.1.1.133.4965}
%%%%%%%%%%%%%%%%%%%%%%%%%%%%%%
\bibitem{plumtree} 
João Leitao, José Pereira, and Luís Rodrigues. 
\textit{Plumtree}: Epidemic broadcast trees.
\\\texttt{https://www.gsd.inesc-id.pt/~ler/reports/srds07.pdf}
%%%%%%%%%%%%%%%%%%%%%%%%%%%%%%
\bibitem{babel} 
\textit{Babel}
\\\texttt{https://asc.di.fct.unl.pt/~jleitao/babel/help-doc.html}
%%%%%%%%%%%%%%%%%%%%%%%%%%%%%%
\bibitem{leitao}
João Carlos Antunes Leitão.
Dissertação de Mestrado.
\\\texttt{https://asc.di.fct.unl.pt/~jleitao/pdf/masterthesis-leitao.pdf}
%%%%%%%%%%%%%%%%%%%%%%%%%%%%%%
\bibitem{cluster}
Cluster DI.
\\\texttt{https://cluster.di.fct.unl.pt/}
%%%%%%%%%%%%%%%%%%%%%%%%%%%%%%

%%%%%%%%%%%%%%%%%%%%%%%%%%%%%%

%%%%%%%%%%%%%%%%%%%%%%%%%%%%%%

%%%%%%%%%%%%%%%%%%%%%%%%%%%%%%

%%%%%%%%%%%%%%%%%%%%%%%%%%%%%%
\end{thebibliography}

\newpage

\begin{appendices}

\lstset{language=Python}

\subsection{Configuration file}
    \begin{lstlisting}
        port=10000
        interface=eth0
        address=127.0.0.1
        sample_size=3
        sample_time=3000
        protocol_metrics_interval=1000
        channel_metrics_interval=1000
        payload_size={PAYLOAD_SIZE}
        prepare_time=5
        run_time=60
        cooldown_time=5
        broadcast_interval={INTERVAL_RATE}
        broadcast={BROADCAST_ALG}
        membership={MEMBERSHIP_ALG}
        activeMembershipSize=5
        passiveMembershipSize=30
        shuffleTime=15000
        joinTime=1000
        k=6
        c=1
        arwl=6
        prwl=3
        ka=3
        kp=4
        n=100
    \end{lstlisting}
    
\subsection{Docker info}
    \begin{lstlisting}
        Client:
         Debug Mode: false
        Server:
         Containers: 0
          Running: 0
          Paused: 0
          Stopped: 0
         Images: 380
         Server Version: 19.03.6
         Storage Driver: overlay2
          Backing Filesystem: extfs
          Supports d_type: true
          Native Overlay Diff: true
         Logging Driver: json-file
         Cgroup Driver: cgroupfs
         Plugins:
          Volume: local
          Network: bridge host ipvlan macvlan null overlay
          Log: awslogs fluentd gcplogs gelf journald json-file local logentries splunk syslog
         Swarm: active
          NodeID: wpj20zepewpo13plsneygvk03
          Error: rpc error: code = Unknown desc = The swarm does not have a leader. It's possible that too few managers are online. Make sure more than half of the managers are online.
          Is Manager: true
          Node Address: 172.30.10.108
          Manager Addresses:
           172.30.10.107:2377
           172.30.10.108:2377
         Runtimes: runc
         Default Runtime: runc
         Init Binary: docker-init
         containerd version:
         runc version:
         init version:
         Security Options:
          apparmor
          seccomp
           Profile: default
         Kernel Version: 5.3.0-64-generic
         Operating System: Ubuntu 19.10
         OSType: linux
         Architecture: x86_64
         CPUs: 24
         Total Memory: 62.87GiB
         Name: node8
         ID:
          XFNZ:7CU6:2VDG:M4K7:XRMR:WE6A:63WW:FORC:VE2V:MMRA:LDBI:KNVE
         Docker Root Dir: /var/lib/docker
         Debug Mode: false
         Registry: https://index.docker.io/v1/
         Labels:
         Experimental: false
         Insecure Registries:
          127.0.0.0/8
         Live Restore Enabled: false
     \end{lstlisting}
     
\end{appendices}

\end{document}
\endinput
%%
%% End of file `sample-sigconf.tex'.
