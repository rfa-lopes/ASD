%%
%% This is file `sample-sigconf.tex',
%% generated with the docstrip utility.
%%
%% The original source files were:
%%
%% samples.dtx  (with options: `sigconf')
%% 
%% IMPORTANT NOTICE:
%% 
%% For the copyright see the source file.
%% 
%% Any modified versions of this file must be renamed
%% with new filenames distinct from sample-sigconf.tex.
%% 
%% For distribution of the original source see the terms
%% for copying and modification in the file samples.dtx.
%% 
%% This generated file may be distributed as long as the
%% original source files, as listed above, are part of the
%% same distribution. (The sources need not necessarily be
%% in the same archive or directory.)
%%
%% The first command in your LaTeX source must be the \documentclass command.
\documentclass[sigconf]{acmart}

%%
%% \BibTeX command to typeset BibTeX logo in the docs
\AtBeginDocument{%
  \providecommand\BibTeX{{%
    \normalfont B\kern-0.5em{\scshape i\kern-0.25em b}\kern-0.8em\TeX}}}

%% Rights management information.  This information is sent to you
%% when you complete the rights form.  These commands have SAMPLE
%% values in them; it is your responsibility as an author to replace
%% the commands and values with those provided to you when you
%% complete the rights form.
\setcopyright{acmcopyright}
\copyrightyear{2020}
\acmYear{2020}
%\acmDOI{10.1145/1122445.1122456}

%% These commands are for a PROCEEDINGS abstract or paper.
\acmConference[ASD20/21]{The first project delivery of ASD2021}{2020}{Faculdade de Ciências e Tecnologia, NOVA University of Lisbon, Portugal}
\acmBooktitle{The Projects of ASD - first delivery, 2020, Faculdade de Ciências e Tecnologia, NOVA University of Lisbon, Portugal}
%\acmPrice{15.00}
%\acmISBN{978-1-4503-XXXX-X/18/06}


%%
%% Submission ID.
%% Use this when submitting an article to a sponsored event. You'll
%% receive a unique submission ID from the organizers
%% of the event, and this ID should be used as the parameter to this command.
%%\acmSubmissionID{123-A56-BU3}

%%
%% The majority of ACM publications use numbered citations and
%% references.  The command \citestyle{authoryear} switches to the
%% "author year" style.
%%
%% If you are preparing content for an event
%% sponsored by ACM SIGGRAPH, you must use the "author year" style of
%% citations and references.
%% Uncommenting
%% the next command will enable that style.
%%\citestyle{acmauthoryear}

%%
%% end of the preamble, start of the body of the document source.
\begin{document}

%%
%% The "title" command has an optional parameter,
%% allowing the author to define a "short title" to be used in page headers.
\title{Implementation and Experimental Measurements of Membership and Broadcast Algorithms}

%%
%% The "author" command and its associated commands are used to define
%% the authors and their affiliations.
%% Of note is the shared affiliation of the first two authors, and the
%% "authornote" and "authornotemark" commands
%% used to denote shared contribution to the research.
\author{Rodrigo Faria Lopes}
\authornote{Student number 50435. Rodrigo was responsible for implementing HyParView.}
\email{rfa.lopes@campus.fct.unl.pt}
\affiliation{%
  \institution{MIEI, DI, FCT, UNL}
}

\author{Miguel Candeias}
\authornote{Student number 50647. Miguel was responsible for implementing Cyclon.}
\email{mb.candeias@campus.fct.unl.pt}
\affiliation{%
  \institution{MIEI, DI, FCT, UNL}
}

\author{Salvador Rosa Mendes}
\authornote{Student number 50503. Salvador was responsible for implementing Plumtree.}
\email{sr.mendes@campus.fct.unl.pt}
\affiliation{%
  \institution{MIEI, DI, FCT, UNL}
}

%%
%% By default, the full list of authors will be used in the page
%% headers. Often, this list is too long, and will overlap
%% other information printed in the page headers. This command allows
%% the author to define a more concise list
%% of authors' names for this purpose.
\renewcommand{\shortauthors}{Lopes, Candeias, and Mendes.}

%%
%% The abstract is a short summary of the work to be presented in the
%% article.
\begin{abstract}
  TODO
  
  Primeiro paragrafo - Contexto e objectivos
  
  Segundo paragrafo - O que fizemos em concreto e resultados obtidos
\end{abstract}


%%
%% This command processes the author and affiliation and title
%% information and builds the first part of the formatted document.
\maketitle

\section{Introduction}
TODO
(4 ou 5 paragrafos)
Contexto - Porque é que isto é interessante?
Contexto muito geral (paragrafo)
Contexto menos geral (paragrafo)
Contexto especifico (paragrafo)

(paragrafo)
Estrutura do documento

\section{Related Work}
TODO - Explicar os algoritmos

2 Subseccoes
    1-Overlay protocols
    2-Broadcast gossip
    3(opcional) - Discutir outros estudos praticos destas soluções (poque é que o nosso é melhor?)

\section{Implementation}
TODO - Explicar a implementação e possíveis modificações
pseudocodigo, etc

paragrafo - Indice desta secção

2 subseccoes


\section{Experimental Evaluation}
TODO

Detalhes de forma a que se alguem quiser replicar os testes, consiga:
-Tipo de hardware(caractristicas)
-parametros dos protocolos
-Tempos
-Processamento às medidas?

metodologia

resultados (gráficos melhor que tabelas)
gráfico único com várias linhas

discussão dos resultados um a um
-Explicar o que esta no gráfico
-Extrair conclusões

\section{Conclusions}
%TODO

%sumário do que apresentamos

%future work

%limitações que o nosso trabalho tem

Concluindo este relatório, através do desenvolvimento deste projecto foi possível aplicar os conhecimentos teóricos adquiridos nas aulas, pôr os mesmos em prática e testa-los numa situação real como no \textit{cluster} do Departamento de Informática.

Os protocolos à base de gossip tem um papel fundamental, no sentido em que deixa de ser necessário fazer \textit{flood} na rede para assegurar que todos os processos nesta, receberam efectivamente a mensagem. Sendo assim este tipo de protocolos, baseia-se na evidência de que é apenas necessário enviar um reduzido numero de mensagens para assegurar a disseminação efectiva de uma mensagem na rede, aumentando assim a disponibilidade e fazendo menos pressão no envio de mensagens num processo.

Relativamente aos protocolos de \textit{Membership}, estes desempenham um papel fundamental, pois tem como papel principal a manutenção e organização da \textit{network overlay} que liga todos os processos, com o objectivo de facultar ao algoritmo de \textit{broadcast} uma vista parcial da rede para ser possível a disseminação de mensagens.

Foi ainda possível concluir, pelos dados estatísticos, que existem certas combinações de protocolos que funcionam melhor do que outras, e que, por exemplo quando o \textit{payload} era aumentado havia uma condensação/pressão maior na rede, levando assim a uma disponibilidade menor, o que é suportado pelos dados estatísticos. 

O \textit{bitrate} é ainda outro factor que é determinante na performance dos testes, a escolha dos diferentes \textit{bitrates} utilizados no teste, não foi a mais adequada visto que a diferença entre eles era ínfima.

Ainda assim, houve algumas limitações na implementação dos algoritmos, levando assim a alguma dificuldade na extracção dos dados estatísticos, mas futuramente seria algo que poderia ser melhorado, como por exemplo no algoritmo EagerPush, relativamente ao tempo em que este devia guardar as mensagens, por razões de complexidade espacial e finalmente aumentar o número de testes utilizando outras combinações de parâmetros com o fim de aumentar a diversidade das métricas avaliadas.


\bibliographystyle{ACM-Reference-Format}

\begin{thebibliography}{9}

%%%%%%%%%%%%%%%%%%%%%%%%%%%%%%
\bibitem{hyparview} 
João Leitão, José Pereira, and Luís Rodrigues. 
\textit{HyParView}: a membership protocol for reliable gossip-based broadcast.
\\\texttt{https://asc.di.fct.unl.pt/~jleitao/pdf/dsn07-leitao.pdf}
%%%%%%%%%%%%%%%%%%%%%%%%%%%%%%
\bibitem{cyclon} 
Spyros Voulgaris, Daniela Gavidia, and Maarten van Steen. 
\textit{Cyclon}: Inexpensive membership management for unstructured p2p overlays.
\\\texttt{http://citeseerx.ist.psu.edu/viewdoc/summary?doi=10.1.1.133.4965}
%%%%%%%%%%%%%%%%%%%%%%%%%%%%%%
\bibitem{plumtree} 
João Leitao, José Pereira, and Luís Rodrigues. 
\textit{Plumtree}: Epidemic broadcast trees.
\\\texttt{https://www.gsd.inesc-id.pt/~ler/reports/srds07.pdf}
%%%%%%%%%%%%%%%%%%%%%%%%%%%%%%
\bibitem{babel} 
\textit{Babel}
\\\texttt{https://asc.di.fct.unl.pt/~jleitao/babel/help-doc.html}
%%%%%%%%%%%%%%%%%%%%%%%%%%%%%%
\bibitem{leitao}
João Carlos Antunes Leitão.
Dissertação de Mestrado.
\\\texttt{https://asc.di.fct.unl.pt/~jleitao/pdf/masterthesis-leitao.pdf}
%%%%%%%%%%%%%%%%%%%%%%%%%%%%%%
\bibitem{cluster}
Cluster DI.
\\\texttt{https://cluster.di.fct.unl.pt/}
%%%%%%%%%%%%%%%%%%%%%%%%%%%%%%

%%%%%%%%%%%%%%%%%%%%%%%%%%%%%%

%%%%%%%%%%%%%%%%%%%%%%%%%%%%%%

%%%%%%%%%%%%%%%%%%%%%%%%%%%%%%

%%%%%%%%%%%%%%%%%%%%%%%%%%%%%%
\end{thebibliography}

\end{document}
\endinput
%%
%% End of file `sample-sigconf.tex'.
